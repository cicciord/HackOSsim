\section{Introduction}
The HacklOSsim project presents the opportunity for exploration and customization of an embedded operation system within the context of an ISA simulator.
The chosen simulator, QEMU (Quick Emulator), will be the bridge that allows us to bring our embedded system to life virtually.

These are the primary objectives of the HaclOSsim project, with our chosen operating system being FreeRTOS:

 1)Mastery of QEMU: We begin with the development of proficiency in using QEMU as our ISA simulator. In addition to run FreeRTOS within this environment but also create a comprehensive tutorial that outlines the installation and usage procedures. This foundational knowledge will serve as a stepping stone for the subsequent tasks.
  
 2)Practical Exploration: In this point we develop practical examples and exercises that vividly illustrate the functionality of the FreeRTOS operating system within the QEMU simulator. These exercises will be designed to align with the topics studied in our coursework, ensuring that theory meets practice seamlessly.
  
 3)Customization for Enhanced Functionality: The main part of this project lies in customizing the FreeRTOS operating system to implement a new, straightforward memory management system, by applying our theoretical knowledge to practical scenarios.
  
 4)Performance Evaluation: In order to quantify the impact of our customization, we will conduct rigorous performance evaluations and benchmarking. These evaluations will offer insights into the real-world efficiency and effectiveness of the new feature within FreeRTOS. It will provide empirical evidence of the enhancements achieved.







