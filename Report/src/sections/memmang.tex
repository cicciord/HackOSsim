\section{Memory Management}

A custom memory management library has been implemented. The library is a modified version of the \texttt{heap\_4.c} provided by FreeRTOS.

In the custom implementation it has been developed the \textit{realloc} function, among with \textit{best-fit} and \textit{worst-fit} allocation strategy.

\subsection{Realloc}
    \subsubsection{Implementation}
    The implementation of the realloc function is the following:

    \begin{lstlisting}[language=c]
    void *pvPortRealloc(void *pv, size_t xWantedSize);
    \end{lstlisting}

    The function takes two arguments, the pointer to the memory block to be reallocated and the new size of the memory block. The function returns a pointer to the reallocated memory block.

    In case a \texttt{NULL} pointer is passed as the first argument, the function behaves as a malloc call. In case the second argument is 0, the function behaves as a free call.
    
    In any other case the funcion will try to reallocate the memory block, after checking that the pointer corresponds to a valid memory block.

    The reallocation is implemented in the following way:

    \begin{codebox}
    \begin{lstlisting}
    BlockLink_t * pxLink;
    void * pvReturn = NULL;
    uint8_t * puc = ( uint8_t * ) pv;
    size_t xBlockSize;
    size_t xMoveSize;

    /* The memory being reallocated will have an BlockLink_t structure immediately
        * before it. */
    puc -= xHeapStructSize;

    /* This casting is to keep the compiler from issuing warnings. */
    pxLink = ( void * ) puc;
    \end{lstlisting}
    \end{codebox}

    \begin{codebox}
    \begin{lstlisting}[language=c]
    /* The size of the block being reallocated */
    xBlockSize = ( pxLink->xBlockSize & ~heapBLOCK_ALLOCATED_BITMASK ) - xHeapStructSize;

    /* Allocate new memory block */
    pvReturn = pvPortMalloc( xWantedSize );

    if( pvReturn != NULL )
    {
        /* The size of the memory to copy */
        xMoveSize = xBlockSize < xWantedSize ? xBlockSize : xWantedSize;

        /* Copy the memory to the new location */
        memcpy( pvReturn, pv, xMoveSize );

        /* Free old memory block */
        vPortFree(pv);
    } 
    \end{lstlisting}
    \end{codebox}

    Notice that FreeRTOS use the first bit of the block size to store the allocation status of the block. For this reason if a block is allocated the size of the block is obtained by clearing the first bit of the block size.

    \subsubsection{Testing}
    The realloc function has been tested on the following cases:

    \paragraph{Reallocating a block to a larger size}
    To test the realloc function a block of memory of 10 bytes is allocated and then reallocated to a size of 30 bytes. It can be observed that the memory is reallocated to a new address and the old memory is freed.

    \paragraph{Reallocating a block to a smaller size}
    The test is similar to the previus one but with initial size of 30 bytes and reallocation to 10 bytes. It can be observed that the memory is reallocated to a new address and the old memory is freed. Only the first 10 bytes of the old memory are copied to the new memory.

    \paragraph{Reallocating a NULL pointer}
    The \texttt{NULL} pointer is reallocated, it can be observed that the memory is allocated to a new address, same as a malloc call.

    \paragraph{Reallocating a block to size 0}
    A block is reallocated to size 0, it can be observed that the memory is freed, same as a free call.

\subsection{Best-Fit and Worst-fit Strategy}
    \subsubsection{Implementation}
    FreeRTOS by default use a \textit{first-fit} allocation strategy and handles the free blocks in a linked list. In order to allocate a block the list is traversed until a block greater than the requested size is found. 

    The \textit{best-fit} strategy is implemented by traversing the list and keeping track of the smallest block greater than the requested size. The first block greater than the requested size is returned.

    Following the implementation

    \paragraph{Best-fit} Traverse the list and find the smallest difference between the block size and the requested size.

    \begin{codebox}
    \begin{lstlisting}
#if (configHEAP_ALLOCATION_TYPE == 1)
    configASSERT( heapSUBTRACT_WILL_UNDERFLOW( pxBlock->xBlockSize, xWantedSize ) == 0 );
    /* traverse the whole free block list */
    while( pxBlock->pxNextFreeBlock != heapPROTECT_BLOCK_POINTER( NULL ) )
    {
        /* Check if the current block is a valid option and if another valid block
            was found before and check wheter is a best fit */
        if  (   ( pxBlock->xBlockSize >= xWantedSize )
                &&
                (   pxBlockTmp == NULL
                    ||
                    ( ( pxBlock->xBlockSize - xWantedSize ) < ( pxBlockTmp->xBlockSize - xWantedSize ) )
                )
            )
        {
            pxPreviousBlockTmp = pxPreviousBlock;
            pxBlockTmp = pxBlock;
        }
        pxPreviousBlock = pxBlock;
        pxBlock = heapPROTECT_BLOCK_POINTER( pxBlock->pxNextFreeBlock );
        heapVALIDATE_BLOCK_POINTER( pxBlock );
    }
    pxPreviousBlock = pxPreviousBlockTmp;
    pxBlock = pxBlockTmp;
    \end{lstlisting}
    \end{codebox}

    \paragraph{Worst-fit} Traverse the list and find the largest difference between the block size and the requested size.

    \begin{codebox}
    \begin{lstlisting}
#elif (configHEAP_ALLOCATION_TYPE == 2)
    configASSERT( heapSUBTRACT_WILL_UNDERFLOW( pxBlock->xBlockSize, xWantedSize ) == 0 );
    /* traverse the whole free block list */
    while( pxBlock->pxNextFreeBlock != heapPROTECT_BLOCK_POINTER( NULL ) )
    {
        /* Check if the current block is a valid option and if another valid block
            was found before and check wheter is a worst fit */
        if  (   ( pxBlock->xBlockSize >= xWantedSize )
                &&
                (   pxBlockTmp == NULL
                    ||
                    ( ( pxBlock->xBlockSize - xWantedSize ) > ( pxBlockTmp->xBlockSize - xWantedSize ) )
                )
            )
        {
            pxPreviousBlockTmp = pxPreviousBlock;
            pxBlockTmp = pxBlock;
        }
        pxPreviousBlock = pxBlock;
        pxBlock = heapPROTECT_BLOCK_POINTER( pxBlock->pxNextFreeBlock );
        heapVALIDATE_BLOCK_POINTER( pxBlock );
    }
    pxPreviousBlock = pxPreviousBlockTmp;
    pxBlock = pxBlockTmp;
    \end{lstlisting}
    \end{codebox}

    \paragraph{First-fit} Traverse the list and return the first block greater than the requested size. Default case.

    \begin{codebox}
    \begin{lstlisting}
#else
    while( ( pxBlock->xBlockSize < xWantedSize ) && ( pxBlock->pxNextFreeBlock != heapPROTECT_BLOCK_POINTER( NULL ) ) )
    {
        pxPreviousBlock = pxBlock;
        pxBlock = heapPROTECT_BLOCK_POINTER( pxBlock->pxNextFreeBlock );
        heapVALIDATE_BLOCK_POINTER( pxBlock );
    }
#endif
    \end{lstlisting}
    \end{codebox}

    \subsubsection{Testing}
    \paragraph{Best-fit} The best-fit strategy has been tested by creating a memory layout with 5 blocks of memory as follow:

    \begin{table}[h]
    \centering
    \begin{tabular}{|c|c|c|c|c|c|}
    \hline
    & \textbf{Block 1} & \textbf{Block 2} & \textbf{Block 3} & \textbf{Block 4} & \textbf{Block 5} \\
    \hline
    \textbf{Allocated} & No & Yes & No & Yes & No \\
    \textbf{Size (Bytes)} & 64 & 32 & 32 & 32 & 2856 \\
    \hline
    \end{tabular}
    \end{table}

    A new block of 32 bytes is requested, the best-fit strategy should return the block 3, the smallest block greater than the requested size.

    \paragraph{Worst-fit} The worst-fit strategy has been tested by creating a memory layout with 3 blocks of memory as follow:

    \begin{table}[h]
    \centering
    \begin{tabular}{|c|c|c|c|c|c|}
    \hline
    & \textbf{Block 1} & \textbf{Block 2} & \textbf{Block 3} \\
    \hline
    \textbf{Allocated} & No & Yes & No \\
    \textbf{Size (Bytes)} & 64 & 32 & 2936 \\
    \hline
    \end{tabular}
    \end{table}

    A new block of 32 bytes is requested, the worst-fit strategy should return the block 3, the largest block greater than the requested size.