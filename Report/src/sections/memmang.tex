\section{Memory Management}

A custom memory management library has been implemented. The library is a modified version of the \textit{heap\_4.c} implementation provided by FreeRTOS.

In the provided implementation it has been implemented the \textit{realloc} function, among with \textit{best-fit} and \textit{worst-fit} allocation strategy.

\subsection{Realloc}
    \subsubsection{Implementation}
    The implementation of the realloc function is the following.

    \begin{lstlisting}[language=c,breaklines]
    void *pvPortRealloc(void *pv, size_t xWantedSize);
    \end{lstlisting}

    The function takes two arguments, the pointer to the memory block to be reallocated and the new size of the memory block. The function returns a pointer to the reallocated memory block.

    In case a NULL pointer is passed as the first argument, the function behaves as a malloc call. In case the second argument is 0, the function behaves as a free call.
    
    In any other case the funcion will try to reallocate the memory block, after checking that the pointer corresponds to a valid memory block.

    The reallocation is implemented in the following way:

    \begin{codebox}
    \begin{lstlisting}
    BlockLink_t * pxLink;
    void * pvReturn = NULL;
    uint8_t * puc = ( uint8_t * ) pv;
    size_t xBlockSize;
    size_t xMoveSize;

    /* The memory being reallocated will have an BlockLink_t structure immediately
        * before it. */
    puc -= xHeapStructSize;

    /* This casting is to keep the compiler from issuing warnings. */
    pxLink = ( void * ) puc;
    \end{lstlisting}
    \end{codebox}

    \begin{codebox}
    \begin{lstlisting}[language=c]
    /* The size of the block being reallocated */
    xBlockSize = ( pxLink->xBlockSize & ~heapBLOCK_ALLOCATED_BITMASK ) - xHeapStructSize;

    /* Allocate new memory block */
    pvReturn = pvPortMalloc( xWantedSize );

    if( pvReturn != NULL )
    {
        /* The size of the memory to copy */
        xMoveSize = xBlockSize < xWantedSize ? xBlockSize : xWantedSize;

        /* Copy the memory to the new location */
        memcpy( pvReturn, pv, xMoveSize );

        /* Free old memory block */
        vPortFree(pv);
    } 
    \end{lstlisting}
    \end{codebox}

    Notice that FreeRTOS use the first bit of the block size to store the allocation status of the block. For this reason if a block is allocated the size of the block is obtained by clearing the first bit of the block size.

    \subsubsection{Testing}
    The realloc function has been tested on the following cases:

    \begin{itemize}
        \item Reallocating a block to a larger size
        \item Reallocating a block to a smaller size
        \item Reallocating a NULL pointer
        \item Reallocating a block to size 0
    \end{itemize}

    In all test the \textit{configHEAP\_CLEAR\_MEMORY\_ON\_FREE} has been set to 1 to clear the memory after a free operation in order to better visualize the memory status.

    \paragraph{Reallocating a block to a larger size}
    To test the realloc function a block of memory of 10 bytes is allocated and then reallocated to a size of 30 bytes. The memory status is printed before and after the reallocation.

    Following the output of the test

    \begin{outputbox}
    \begin{lstlisting}[language=bash]
    ************************************************************
    *  [TEST]   Reallocate memory to a greater size            *
    ************************************************************
    
    ---[DEBUG]--- Malloc is called with wantedSize 10 bytes
    ---[DEBUG]--- Buffer initialized with 10 char (a)
    
    Initial Block stats:
        Allocated     : Yes
        Block pointer : 0x200005e0
        Data pointer  : 0x200005e8
        Block size    : 24 (bytes)
        Data size     : 16 (bytes)
    
    Block Memory:
       offset b0 b1 b2 b3 b4 b5 b6 b7   ASCII
        0000  00 00 00 00 18 00 00 80  ........
        0008  61 61 61 61 61 61 61 61  aaaaaaaa
        0010  61 61 00 00 00 00 00 00  aa......
    \end{lstlisting}
    \end{outputbox}
    
    \begin{outputbox}
    \begin{lstlisting}[language=bash]
    ---[DEBUG]--- Memory is reallocated with new size 30
    
    New Block stats:
        Allocated     : Yes
        Block pointer : 0x200005f8
        Data pointer  : 0x20000600
        Block size    : 40 (bytes)
        Data size     : 32 (bytes)
    
    Block Memory:
       offset b0 b1 b2 b3 b4 b5 b6 b7   ASCII
        0000  00 00 00 00 28 00 00 80  ....(...
        0008  61 61 61 61 61 61 61 61  aaaaaaaa
        0010  61 61 00 00 00 00 00 00  aa......
        0018  00 00 00 00 00 00 00 00  ........
        0020  00 00 00 00 00 00 00 00  ........
    
    Old Block stats:
        Allocated     : No
        Block pointer : 0x200005e0
        Data pointer  : 0x200005e8
        Block size    : 24 (bytes)
        Data size     : 16 (bytes)
    
    Old Block Memory:
       offset b0 b1 b2 b3 b4 b5 b6 b7   ASCII
        0000  20 06 00 20 18 00 00 00   .. ....
        0008  00 00 00 00 00 00 00 00  ........
        0010  00 00 00 00 00 00 00 00  ........
    
    Heap Stats:
        Total Heap Size                   : 3072
        Available Space in bytes          : 3016
        Largest free block in bytes       : 2992
        Smallest free block in bytes      : 24
        Number of free block              : 2
        Minumum Ever Free Bytes Remaining : 2992
        Successful allocations            : 2
        Successful frees                  : 1
    
    ---[DEBUG]--- Used memory is freed
    
    ************************************************************
    \end{lstlisting}
    \end{outputbox}

    \textbf{Notice}: the cortex-m3 architecture is little-endian, so the memory is printed in reverse order.

    An allocated block of memory can be recognized by the first 8 bytes. The first 4 bytes are the pointer to the next block and the second 4 bytes are the block size. For an allocated block the first 4 bytes are zeros, while the highest bit of the secnd 4 bytes is 1, thus the highest byte is 0x80.

    \paragraph{Reallocating a block to a smaller size}
    The test is similar to the previus one but the block is reallocated to a smaller size.

    Following the output of the test

    \begin{outputbox}
    \begin{lstlisting}[language=bash]
    ************************************************************
    *  [TEST]   Reallocate memory to a smaller size            *
    ************************************************************

    ---[DEBUG]--- Malloc is called with wantedSize 30 bytes
    ---[DEBUG]--- Buffer initialized with 30 char (b)

    Initial Block stats:
        Allocated     : Yes
        Block pointer : 0x200005e0
        Data pointer  : 0x200005e8
        Block size    : 40 (bytes)
        Data size     : 32 (bytes)

    Block Memory:
       offset b0 b1 b2 b3 b4 b5 b6 b7   ASCII
        0000  00 00 00 00 28 00 00 80  ....(...
        0008  62 62 62 62 62 62 62 62  bbbbbbbb
        0010  62 62 62 62 62 62 62 62  bbbbbbbb
        0018  62 62 62 62 62 62 62 62  bbbbbbbb
        0020  62 62 62 62 62 62 00 00  bbbbbb..
    \end{lstlisting}
    \end{outputbox}

    \begin{outputbox}
    \begin{lstlisting}[language=bash]
    ---[DEBUG]--- Memory is reallocated with new size 10

    New Block stats:
        Allocated     : Yes
        Block pointer : 0x20000608
        Data pointer  : 0x20000610
        Block size    : 24 (bytes)
        Data size     : 16 (bytes)

    Block Memory:
       offset b0 b1 b2 b3 b4 b5 b6 b7   ASCII
        0000  00 00 00 00 18 00 00 80  ........
        0008  62 62 62 62 62 62 62 62  bbbbbbbb
        0010  62 62 00 00 00 00 00 00  bb......

    Old Block stats:
        Allocated     : No
        Block pointer : 0x200005e0
        Data pointer  : 0x200005e8
        Block size    : 40 (bytes)
        Data size     : 32 (bytes)

    Old Block Memory:
       offset b0 b1 b2 b3 b4 b5 b6 b7   ASCII
        0000  20 06 00 20 28 00 00 00   .. (...
        0008  00 00 00 00 00 00 00 00  ........
        0010  00 00 00 00 00 00 00 00  ........
        0018  00 00 00 00 00 00 00 00  ........
        0020  00 00 00 00 00 00 00 00  ........

    Heap Stats:
        Total Heap Size                   : 3072
        Available Space in bytes          : 3032
        Largest free block in bytes       : 2992
        Smallest free block in bytes      : 40
        Number of free block              : 2
        Minumum Ever Free Bytes Remaining : 2992
        Successful allocations            : 4
        Successful frees                  : 3

    ---[DEBUG]--- Used memory is freed

    ************************************************************
    \end{lstlisting}
    \end{outputbox}

    \paragraph{Reallocating a NULL pointer}
    The NULL pointer is reallocated, it can be observed that the memory is allocated to a new address, same as a malloc call.

    \begin{outputbox}
    \begin{lstlisting}[language=bash]
    ************************************************************
    *  [TEST]   Allocate memory using realloc                  *
    ************************************************************

    ---[DEBUG]--- Realloc is called with wantedSize 10 bytes and
                  NULL pointer
    ---[DEBUG]--- Buffer initialized with 10 char (c)
    
    Block stats:
      Allocated     : Yes
      Block pointer : 0x200005e0
      Data pointer  : 0x200005e8
      Block size    : 24 (bytes)
      Data size     : 16 (bytes)
    
    Block Memory:
     offset b0 b1 b2 b3 b4 b5 b6 b7   ASCII
      0000  00 00 00 00 18 00 00 80  ........
      0008  63 63 63 63 63 63 63 63  cccccccc
      0010  63 63 00 00 00 00 00 00  cc......
    
    Heap Stats:
      Total Heap Size                   : 3072
      Available Space in bytes          : 3032
      Largest free block in bytes       : 3032
      Smallest free block in bytes      : 3032
      Number of free block              : 1
      Minumum Ever Free Bytes Remaining : 2992
      Successful allocations            : 5
      Successful frees                  : 4
    
    ---[DEBUG]--- Used memory is freed
    
    ************************************************************
    \end{lstlisting}
    \end{outputbox}

    \paragraph{Reallocating a block to size 0}
    The block is reallocated to size 0, it can be observed that the memory is freed, same as a free call.

    \begin{outputbox}
    \begin{lstlisting}[language=bash]
    ************************************************************
    *  [TEST]   Free memory uising realloc                     *
    ************************************************************
        
    ---[DEBUG]--- Malloc is called with wantedSize 10
    ---[DEBUG]--- Buffer initialized with 10 char (d)
    
    Block stats:
        Allocated     : Yes
        Block pointer : 0x200005e0
        Data pointer  : 0x200005e8
        Block size    : 24 (bytes)
        Data size     : 16 (bytes)
    
    Block Memory:
       offset b0 b1 b2 b3 b4 b5 b6 b7   ASCII
        0000  00 00 00 00 18 00 00 80  ........
        0008  64 64 64 64 64 64 64 64  dddddddd
        0010  64 64 00 00 00 00 00 00  dd......
    
    ---[DEBUG]--- Memory is reallocated with new size 0
    
    Block stats:
        Allocated     : No
        Block pointer : 0x200005e0
        Data pointer  : 0x200005e8
        Block size    : 3056 (bytes)
        Data size     : 3048 (bytes)
    
    Block Memory:
       offset b0 b1 b2 b3 b4 b5 b6 b7   ASCII
        0000  d0 11 00 20 f0 0b 00 00  ... ....
        0008  00 00 00 00 00 00 00 00  ........
    
    Heap Stats:
        Total Heap Size                   : 3072
        Available Space in bytes          : 3056
        Largest free block in bytes       : 3056
        Smallest free block in bytes      : 3056
        Number of free block              : 1
        Minumum Ever Free Bytes Remaining : 2992
        Successful allocations            : 6
        Successful frees                  : 6
    
    ************************************************************
    \end{lstlisting}
    \end{outputbox}

\subsection{Best-Fit and Worst-fit Strategy}
    \subsubsection{Implementation}
    FreeRTOS by default use a \textit{first-fit} allocation strategy and handles the free blocks in a linked list. In order to allocate a block the list is traversed until a block greater than the requested size is found. 

    The \textit{best-fit} strategy is implemented by traversing the list and keeping track of the smallest block greater than the requested size. The first block greater than the requested size is returned.

    Following the implementation

    \paragraph{Best-fit} Traverse the list and find the smallest difference between the block size and the requested size.

    \begin{codebox}
    \begin{lstlisting}
    #if (configHEAP_ALLOCATION_TYPE == 1)
        configASSERT( heapSUBTRACT_WILL_UNDERFLOW( pxBlock->xBlockSize, xWantedSize ) == 0 );
        /* traverse the whole free block list */
        while( pxBlock->pxNextFreeBlock != heapPROTECT_BLOCK_POINTER( NULL ) )
        {
            /* Check if the current block is a valid option and if another valid block
               was found before and check wheter is a best fit */
            if  (   ( pxBlock->xBlockSize >= xWantedSize )
                    &&
                    (   pxBlockTmp == NULL
                        ||
                        ( ( pxBlock->xBlockSize - xWantedSize ) < ( pxBlockTmp->xBlockSize - xWantedSize ) )
                    )
                )
            {
                pxPreviousBlockTmp = pxPreviousBlock;
                pxBlockTmp = pxBlock;
            }
            pxPreviousBlock = pxBlock;
            pxBlock = heapPROTECT_BLOCK_POINTER( pxBlock->pxNextFreeBlock );
            heapVALIDATE_BLOCK_POINTER( pxBlock );
        }
        pxPreviousBlock = pxPreviousBlockTmp;
        pxBlock = pxBlockTmp;
    \end{lstlisting}
    \end{codebox}

    \paragraph{Worst-fit} Traverse the list and find the largest difference between the block size and the requested size.

    \begin{codebox}
    \begin{lstlisting}
    #elif (configHEAP_ALLOCATION_TYPE == 2)
        configASSERT( heapSUBTRACT_WILL_UNDERFLOW( pxBlock->xBlockSize, xWantedSize ) == 0 );
        /* traverse the whole free block list */
        while( pxBlock->pxNextFreeBlock != heapPROTECT_BLOCK_POINTER( NULL ) )
        {
            /* Check if the current block is a valid option and if another valid block
               was found before and check wheter is a worst fit */
            if  (   ( pxBlock->xBlockSize >= xWantedSize )
                    &&
                    (   pxBlockTmp == NULL
                        ||
                        ( ( pxBlock->xBlockSize - xWantedSize ) > ( pxBlockTmp->xBlockSize - xWantedSize ) )
                    )
                )
            {
                pxPreviousBlockTmp = pxPreviousBlock;
                pxBlockTmp = pxBlock;
            }
            pxPreviousBlock = pxBlock;
            pxBlock = heapPROTECT_BLOCK_POINTER( pxBlock->pxNextFreeBlock );
            heapVALIDATE_BLOCK_POINTER( pxBlock );
        }
        pxPreviousBlock = pxPreviousBlockTmp;
        pxBlock = pxBlockTmp;
    \end{lstlisting}
    \end{codebox}

    \paragraph{First-fit} Traverse the list and return the first block greater than the requested size. Default case.

    \begin{codebox}
    \begin{lstlisting}
    #else
        while( ( pxBlock->xBlockSize < xWantedSize ) && ( pxBlock->pxNextFreeBlock != heapPROTECT_BLOCK_POINTER( NULL ) ) )
        {
            pxPreviousBlock = pxBlock;
            pxBlock = heapPROTECT_BLOCK_POINTER( pxBlock->pxNextFreeBlock );
            heapVALIDATE_BLOCK_POINTER( pxBlock );
        }
    #endif
    \end{lstlisting}
    \end{codebox}

    \subsubsection{Testing}
    \paragraph{Best-fit} The best-fit strategy has been tested by creating a memory layout with 5 blocks of memory as follow:

    %table with 6 columns and 3 rows (header row and column included)

    \begin{table}[h]
    \centering
    \begin{tabular}{|c|c|c|c|c|c|}
    \hline
    & \textbf{Block 1} & \textbf{Block 2} & \textbf{Block 3} & \textbf{Block 4} & \textbf{Block 5} \\
    \hline
    \textbf{Allocated} & No & Yes & No & Yes & No \\
    \textbf{Byte} & 64 & 32 & 32 & 32 & 2856 \\
    \hline
    \end{tabular}
    \end{table}

    A new block of 32 bytes is requested, the best-fit strategy should return the block 3, the smallest block greater than the requested size.

    Following the output of the test

    \begin{outputbox}
    \begin{lstlisting}[language=bash]
    ************************************************************
    *  [TEST]   Best-Fit Allocation of a memory block          *
    ************************************************************

    ---[DEBUG]--- A first block is allocated of size 64bytes

    ---[DEBUG]--- A second block is allocated of size 32bytes

    ---[DEBUG]--- A third block is allocated of size 32bytes

    ---[DEBUG]--- A fourth block is allocated of size 32bytes

    ---[DEBUG]--- Free the first block to leave a hole of 64bytes

    ---[DEBUG]--- Free the third block to leave a hole of 32bytes
    \end{lstlisting}
    \end{outputbox}

    \begin{outputbox}
    \begin{lstlisting}[language=bash]
    First Block stats:
        Allocated     : No
        Block pointer : 0x200005e0
        Data pointer  : 0x200005e8
        Block size    : 72 (bytes)
        Data size     : 64 (bytes)
    \end{lstlisting}
    \end{outputbox}

    \begin{outputbox}
    \begin{lstlisting}[language=bash]
    Second Block stats:
        Allocated     : Yes
        Block pointer : 0x20000628
        Data pointer  : 0x20000630
        Block size    : 40 (bytes)
        Data size     : 32 (bytes)
    \end{lstlisting}
    \end{outputbox}

    \begin{outputbox}
    \begin{lstlisting}[language=bash]
    Third Block stats:
        Allocated     : No
        Block pointer : 0x20000650
        Data pointer  : 0x20000658
        Block size    : 40 (bytes)
        Data size     : 32 (bytes)
    \end{lstlisting}
    \end{outputbox}

    \begin{outputbox}
    \begin{lstlisting}[language=bash]
    Fourth Block stats:
        Allocated     : Yes
        Block pointer : 0x20000678
        Data pointer  : 0x20000680
        Block size    : 40 (bytes)
        Data size     : 32 (bytes)
    \end{lstlisting}
    \end{outputbox}

    \begin{outputbox}
    \begin{lstlisting}[language=bash]
    Last Block stats:
        Allocated     : No
        Block pointer : 0x200006a0
        Data pointer  : 0x200006a8
        Block size    : 2864 (bytes)
        Data size     : 2856 (bytes)
    \end{lstlisting}
    \end{outputbox}

    \begin{outputbox}
    \begin{lstlisting}[language=bash]
    ---[DEBUG]--- Allocate a new block of 32 bytes

    Block stats:
        Allocated     : Yes
        Block pointer : 0x20000650
        Data pointer  : 0x20000658
        Block size    : 40 (bytes)
        Data size     : 32 (bytes)

    ************************************************************
    \end{lstlisting}
    \end{outputbox}

    \paragraph{Worst-fit} The worst-fit strategy has been tested by creating a memory layout with 3 blocks of memory as follow:

    \begin{table}[h]
    \centering
    \begin{tabular}{|c|c|c|c|c|c|}
    \hline
    & \textbf{Block 1} & \textbf{Block 2} & \textbf{Block 3} \\
    \hline
    \textbf{Allocated} & No & Yes & No \\
    \textbf{Byte} & 64 & 32 & 2936 \\
    \hline
    \end{tabular}
    \end{table}

    A new block of 32 bytes is requested, the worst-fit strategy should return the block 3, the largest block greater than the requested size.

    Following the output of the test

    \begin{outputbox}
    \begin{lstlisting}[language=bash]
    ************************************************************
    *  [TEST]   Worst-Fit allocation of a memory block         *
    ************************************************************
    
    ---[DEBUG]--- A first block is allocated of size 64bytes
    
    ---[DEBUG]--- A second block is allocated of size 32bytes
    
    ---[DEBUG]--- Free the first block to leave a hole of 64bytes
    \end{lstlisting}
    \end{outputbox}
    
    \begin{outputbox}
    \begin{lstlisting}[language=bash]
    First Block stats:
        Allocated     : No
        Block pointer : 0x200005e0
        Data pointer  : 0x200005e8
        Block size    : 72 (bytes)
        Data size     : 64 (bytes)
    \end{lstlisting}
    \end{outputbox}
    
    \begin{outputbox}
    \begin{lstlisting}[language=bash]
    Second Block stats:
        Allocated     : Yes
        Block pointer : 0x20000628
        Data pointer  : 0x20000630
        Block size    : 40 (bytes)
        Data size     : 32 (bytes)
    \end{lstlisting}
    \end{outputbox}
    
    \begin{outputbox}
    \begin{lstlisting}[language=bash]
    Third Block stats:
        Allocated     : No
        Block pointer : 0x20000650
        Data pointer  : 0x20000658
        Block size    : 2944 (bytes)
        Data size     : 2936 (bytes)
    \end{lstlisting}
    \end{outputbox}
    
    \begin{outputbox}
    \begin{lstlisting}[language=bash]
    ---[DEBUG]--- Allocate a new block of 32 bytes
    
    Block stats:
        Allocated     : Yes
        Block pointer : 0x20000650
        Data pointer  : 0x20000658
        Block size    : 40 (bytes)
        Data size     : 32 (bytes)
    
    ************************************************************
    \end{lstlisting}
    \end{outputbox}