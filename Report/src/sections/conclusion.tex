\section*{Conclusion}

Our project focused on mastering FreeRTOS on QEMU through a tutorial and three exercises. 
Beginning with a LED blink project, we interfaced with microcontroller hardware using the HAL library to achieve a 1000-millisecond blink interval.
Next, we explored task scheduling, adjusting task priorities and preemption settings to observe different behaviors. 
The third exercise showcased FreeRTOS features like timers, semaphores, and queues, exemplified through a producer-consumer scenario, illustrating task synchronization and communication. 
In our final endeavor, we enhanced memory management by modifying the heap4.c implementation,introducing the realloc function and best-fit/worst-fit allocation strategies. 
Practical examples validated the efficacy of these enhancements. Overall, this project equipped us with essential skills in FreeRTOS development, empowering us to navigate real-time operating systems and embedded systems with confidence and proficiency.
Through practical implementations and comprehensive exploration, we've gained invaluable insights and capabilities for future endeavors in embedded systems development.