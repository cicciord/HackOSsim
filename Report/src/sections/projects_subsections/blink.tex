\subsection{LED Blink Project}

This project focuses on the interaction with a microcontroller's hardware, specifically controlling a LED. The goal is to make a LED blink every 1000 milliseconds. To achieve this, we use the HAL (Hardware Abstraction Layer) library provided by STM.

The HAL library is an Application Programming Interface (API) that abstracts the details of the microcontroller's hardware. This means that we can write code that interacts with the hardware in a high-level and portable way. For this project, the HAL library allows us to control the state of a LED.

\subsubsection{LED Control}

To control the LED, we need to be able to change the state of a specific output pin on the microcontroller. This is the pin to which the LED is connected. In hardware terms, turning the LED on means providing voltage to the pin (setting the corresponding bit to 1), while turning the LED off means removing voltage from the pin (setting the corresponding bit to 0).

The HAL library provides a function, \texttt{HAL\_GPIO\_TogglePin}, that allows us to change the state of a pin. This function takes two arguments: the GPIO port and the pin number. For our project, the GPIO port is \texttt{GPIOA} and the pin number is \texttt{LD2\_Pin}.

Here is an example of how this function can be used to change the state of the LED:

\begin{verbatim}
HAL_GPIO_TogglePin(GPIOA, LD2_Pin);
\end{verbatim}

In this way, every time this line of code is executed, the state of the LED will change. If the LED is on, it will turn off. If the LED is off, it will turn on.