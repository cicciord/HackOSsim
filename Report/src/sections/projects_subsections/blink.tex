\subsection{LED Blink Project}

This project is a simple demonstration of a LED blinking on a NUCLEO-F103RB board. The board is simulated on QEMU and the task is managed by FreeRTOS.

\subsubsection{Task Handler Function}

The task handler function, \texttt{vTaskBlinkLED\_handler}, is executed by the created task. It contains a continuous loop that toggles the state of the LED pin every 1000 milliseconds (1 second). This is achieved using the \texttt{HAL\_GPIO\_TogglePin} function and the \texttt{vTaskDelay(1000)} function.

\begin{verbatim}
void vTaskBlinkLED_handler(void *params)
{
  while(1) {
    HAL_GPIO_TogglePin(GPIOA, LD2_Pin);
    vTaskDelay(1000);
  }
}
\end{verbatim}

\subsubsection{HAL\_GPIO\_TogglePin}

The \texttt{HAL\_GPIO\_TogglePin} function is used to toggle the state of the LED pin. It takes two arguments: the GPIO port and the pin number.

\subsubsection{Requirements}

To run the project, the following requirements are needed:
\begin{itemize}
\item xPack QEMU
\item Arm GNU Toolchain
\end{itemize}

\subsubsection{Usage}

Here are the commands to run, compile, debug, and clean the project:

\begin{itemize}
\item Run Simulation: \texttt{make qemu}
\item Compile: \texttt{make}
\item Debug: Run \texttt{make qemu-gdb} and in another terminal session, run \texttt{arm-none-eabi-gdb build/NUCLEO\_F103RB\_FREERTOS\_BLINK\_LED.elf -ex "target remote localhost:1234"}.
\item Clean: \texttt{make clean}
\end{itemize}