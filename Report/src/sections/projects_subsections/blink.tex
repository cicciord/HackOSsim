\subsection{LED Blink Project}

This project focuses on the interaction with a microcontroller's hardware, specifically controlling a LED. The goal is to make a LED blink every 1000 milliseconds. To achieve this, we use the HAL (Hardware Abstraction Layer) library provided by STM.

The HAL library is an Application Programming Interface (API) that abstracts the details of the microcontroller's hardware. This means that we can write code that interacts with the hardware in a high-level and portable way. For this project, the HAL library allows us to control the state of a LED.

\subsubsection{LED Control}

To control the LED, we need to be able to manipulate the state of a specific output pin on the microcontroller. This pin is the one to which the LED is connected. In hardware terms, powering the LED involves supplying voltage to the pin, which is achieved by setting the corresponding bit to 1. Conversely, to turn off the LED, we remove voltage from the pin by setting the corresponding bit to 0.

The HAL library provides a function, \texttt{HAL\_GPIO\_TogglePin}, that allows us to alter the state of a pin. This function takes two arguments: the GPIO port and the pin number. For our project, the GPIO port is \texttt{GPIOA} and the pin number is \texttt{LD2\_Pin}.

The function \texttt{HAL\_GPIO\_TogglePin(GPIOA, LD2\_Pin)} is used to switch the state of the LED:

\begin{verbatim}
HAL_GPIO_TogglePin(GPIOA, LD2_Pin);
\end{verbatim}

This function call acts as a toggle switch for the LED. Each execution of this line of code will invert the current state of the LED. This allows us to create a blinking effect by repeatedly calling this function with a delay in between.