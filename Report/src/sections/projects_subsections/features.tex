\subsection{FreeRTOS Features Project}

The project demonstrates the intricate workings of a real-time operating system (RTOS) using FreeRTOS. It provides a comprehensive understanding of task scheduling, synchronization, and inter-task communication using semaphores and queues, which are fundamental concepts in any RTOS.

\subsubsection{Project Design}

The project is designed around three main components: a timer, a Task TX, and a Task RX. Each of these components plays a crucial role in the operation of the system and is designed to demonstrate a specific aspect of RTOS.

\begin{description}
\item[Timer] The timer is a software timer that is configured to expire periodically every 1000 milliseconds. Upon expiration, it triggers a callback function. The primary role of this callback function is to give a semaphore. Semaphores are used in RTOS for task synchronization. In this case, the semaphore acts as a signal to Task TX that the timer has expired.
\item[Task TX] Task TX is designed to demonstrate how a task can wait for a semaphore and perform an operation once the semaphore is given. In its operation, Task TX waits for the semaphore. When the semaphore is given by the timer's callback function, Task TX takes it and performs its operation: it sends a value to a queue. This value is a counter that is incremented in each loop, serving as a simple demonstration of a task performing work in response to a semaphore.
\item[Task RX] Task RX is designed to demonstrate how tasks can wait for data on a queue and process it. Task RX waits for an item to be available in the queue. When Task TX sends a value to the queue, Task RX receives it. It then prints the received value and the current tick count. This demonstrates a task that consumes data produced by another task and the use of the system tick count to measure time in a real-time system.
\end{description}

//add details about the implementation of the project